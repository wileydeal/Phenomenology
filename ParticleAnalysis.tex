\documentclass[11pt]{article} % use larger type; default would be 10pt

\usepackage[utf8]{inputenc} % set input encoding (not needed with XeLaTeX)
\usepackage{geometry} % to change the page dimensions
\geometry{a4paper} % or letterpaper (US) or a5paper or....
\usepackage{graphicx} % support the \includegraphics command and options
\usepackage{booktabs} % for much better looking tables
\usepackage{array} % for better arrays (eg matrices) in maths
\usepackage{paralist} % very flexible & customisable lists (eg. enumerate/itemize, etc.)
\usepackage{verbatim} % adds environment for commenting out blocks of text & for better verbatim
\usepackage{subfig} % make it possible to include more than one captioned figure/table in a single float
\usepackage{fancyhdr} % This should be set AFTER setting up the page geometry
\pagestyle{fancy} % options: empty , plain , fancy
\renewcommand{\headrulewidth}{0pt} % customise the layout...
\lhead{}\chead{}\rhead{}
\lfoot{}\cfoot{\thepage}\rfoot{}
\usepackage{sectsty}
\allsectionsfont{\sffamily\mdseries\upshape} % (See the fntguide.pdf for font help)
\usepackage[nottoc,notlof,notlot]{tocbibind} % Put the bibliography in the ToC
\usepackage[titles,subfigure]{tocloft} % Alter the style of the Table of Contents
\renewcommand{\cftsecfont}{\rmfamily\mdseries\upshape}
\renewcommand{\cftsecpagefont}{\rmfamily\mdseries\upshape} % No bold!

\title{Physics 835 Utility Code}
\author{Robert Wiley Deal}
%\date{} % Activate to display a given date or no date (if empty),
         % otherwise the current date is printed 

\begin{document}
\maketitle

\section{Configuring Source Control}

First, check to make sure Git is installed. The following article explains how to install on both Mac and Windows.
\newline
https://www.atlassian.com/git/tutorials/install-git
\newline
\newline
It will also be helpful to download the command-line Git cheatsheet:
\newline
https://www.atlassian.com/git/tutorials/atlassian-git-cheatsheet
\newline
\newline
Once Git is installed, we're ready to download the code repository. In Terminal, type the following command:
\newline
git clone https://github.com/wileydeal/Phenomenology
\newline
\newline
Now we need to do a quick configuration:
\newline
git remote add origin https://github.com/wileydeal/Phenomenology
\newline
git config user.name \textless your wisc.edu email here\textgreater
\newline
\newline
Now getting updates is as simple as changing to the repository directory and typing: 
\newline
git pull origin master


\end{document}
